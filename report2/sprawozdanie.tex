\documentclass[10pt,a4paper]{article}
\usepackage[utf8]{inputenc}
\usepackage{polski}
\usepackage{amsmath}
\usepackage{amsfonts}
\usepackage{amssymb}
\usepackage{graphicx}
\usepackage{verbatim}
%\usepackage{minted}
\usepackage{amsmath}
\usepackage{algorithm}
\usepackage[noend]{algpseudocode}
\usepackage{csvsimple}
\usepackage{caption}
\usepackage{listings}     
\usepackage{subcaption}
\usepackage{hyperref}
\usepackage{url}
\usepackage[T1]{fontenc}
\usepackage{lmodern}
\usepackage[most]{tcolorbox}
\usepackage{xcolor}
\usepackage{listings}

\lstdefinestyle{BashInputStyle}{
  language=bash,
  basicstyle=\small\sffamily,
  numbers=left,
  numberstyle=\tiny,
  numbersep=3pt,
  frame=tb,
  columns=fullflexible,
  backgroundcolor=\color{yellow!20},
  linewidth=0.9\linewidth,
  xleftmargin=0.1\linewidth
}

\newcommand*{\Package}[1]{\texttt{#1}}%

\author{Onaszkiewicz Przemysław, Gadawski Łukasz}
\title{Temat 2\\ Dla wskazanego kandydata w wyborach w USA, przedstaw w sposób czytelny na mapie finansowanie jego kampanii wyborczej, z uwzględnieniem darowizn bezpośrednich i poprzez komitety wyborcze. 
}

\begin{document}
\maketitle

\section{Architektura rozwiązania}
Aplikacja składa się następujących elementów:
\begin{enumerate}
\item Baza danych PostgreSQL wraz z rozszerzeniem PostGIS.
\item Instacja serwera \textit{geoserver} umożliwiająca pobieranie danych geograficznych z bazy danych.
\item Zadania zaimplementowane jako tzw. \textit{task}i w systemie budowania wersji gradle:
\begin{itemize}
\item[--] \textit{getData} - umożliwiające pobranie plików CSV zawierających dane numeryczne odpowiednich danych finansowania. Poprzez zmianę skryptu możliwe jest pobranie danych z różnych przedziałów lat.
\item[--] \textit{cleanDb} - wykonuje połączenie z bazą danych oraz wykonanie skryptu tworzącego strukturę bazy danych.
\end{itemize} 
\item Skrypt w języku python przetwarzający pliki CSV z danymi dotyczącymi finansowania i ładującymi odpowiednie dane do bazy danych.
\item Aplikacja internetowa oparta o framework aplikacji internetowych \textit{Express} stworzony pod kątem aplikacji napisanych w Node.js.
\begin{itemize}
\item[--] biblioteka AngualarJS udostępniająca komponenty HTML,
\item[--] biblioteka openlayers umożliwiające prezentację danych pobranych z serwera \textit{geoserver},z
\end{itemize}
\end{enumerate}
\section{Opis instalacji}
Aplikacja była testowana na systemie \textit{Linux Mint 17.3 Cinnamon 64-bit} w wersji \textit{2.8.6}. 

\subsection{Przygotowanie bazy danych}
\bigskip \noindent
Instalacja bazy danych wraz z rozszerzeniem PostGIS oraz sterownika wykorzystywanego przy uruchomieniu skryptu języka python umożliwiającego połączenie z bazą danych (w komendzie zawarte są również wszystkie niezbędne narzędzia potrzebne na kolejnym etapie realizacji projektu):
\begin{lstlisting}[style=BashInputStyle]
  # sudo apt-get update
  # sudo apt-get install postgresql postgresql-contrib 
  	postgis postgresql-9.4-postgis-2.1
  	postgresql-9.4-postgis-scripts 
  	postgresql-9.4-postgis-2.1-scripts
  	python3-psycopg2 install qgis python-qgis qgis-plugin-grass
  	shp2pgsql
\end{lstlisting}

\bigskip \noindent
Przygotowanie bazy danych pod kątek wykorzystania przez serwer \textit{geoserver} oraz aplikację internetową. Stworzenie użytkownika bazy danych:
\begin{lstlisting}[style=BashInputStyle]
  # sudo -i -u postgres
  # createuser --interactive // create "tass-user"
\end{lstlisting}

\bigskip \noindent
Stworzenie bazy danych:
\begin{lstlisting}[style=BashInputStyle]
  # createdb tass
\end{lstlisting}

\bigskip \noindent
Instalacja rozszerzenie PostGIS umożliwiającego wykonywanie operacji na danych geograficznych:
\begin{lstlisting}[style=BashInputStyle]
  # sudo -i -u postgres // login as superuser
  # psql -d tass
  # CREATE EXTENSION postgis;
  # CREATE EXTENSION postgis_topology;
\end{lstlisting}

\bigskip \noindent
Zmiana hasła użytkownika:
\begin{lstlisting}[style=BashInputStyle]
  # psql -d tass
  # ALTER user "tass-user" PASSWORD '1234';
\end{lstlisting}

\bigskip \noindent
Stworzenie struktury bazy danych:
\begin{lstlisting}[style=BashInputStyle]
  # gradle cleanDb
\end{lstlisting}

\subsection{Pobranie oraz przygotowanie danych dotyczących finansowania kandydatów}
\bigskip \noindent
Pobranie danych ze stron FEC (Federal Election Commission):
\begin{lstlisting}[style=BashInputStyle]
  # gradle getData
\end{lstlisting}

\bigskip \noindent
Teraz gdy dane z danego okresu są pobrane należy wykonać skrypt \textit{\textbf{extract\_data.py}}

\subsection{Przygotowanie danych geograficznych}
\bigskip \noindent
Przygotowanie danych geograficznych odbywa się poprze pobranie mapy Stanów Zjednoczonych ze strony ESRI:
\url{http://www.arcgis.com/home/item.html?id=8d2012a2016e484dafaac0451f9aea24}

Dane są dostarczone w formacie GDP, należy je w dalszej kolejności skownertować do postaci SHP (Shapefile) przy użyciu narzędzia QGIS. Należy najpierw wczytać dane w formacie GDP, a następnie tak wczytane dane zapisać w formacie SHP.

Kolejnym krokiem jest zapis danych w formacie SHP do bazy danych PostgreSQL, aby umożliwić wykonywanie zapytań przestrzennych. Odbywa się to użycia narzędzia \textit{shp2pgsql}:
\bigskip \noindent
\begin{lstlisting}[style=BashInputStyle]
  # shp2pgsql -I -s 4326 esri-zip-codes.shp
  	geo_zip_codes | psql -U tass-user -d tass
\end{lstlisting}

W tym momencie dane geograficzne zostały załadowane do bazy danych. Aby ułatwić ich prezentację baza danych zostanie podłączona do serwera \textit{geoserver}, tak aby umożliwić pobranie danych przy pomocy usług WMS oraz WFS. 

Należy zainstalować geoserver zgodnie z poradnikiem ze strony \url{http://docs.geoserver.org/stable/en/user/installation/linux.html} 

Następnie uruchomić serwer poprzez następujące komendy:
\begin{lstlisting}[style=BashInputStyle]
  # cd /usr/share/geoserver/bin
  # ./startup.sh
\end{lstlisting}

Domyślnie panel administracyjny będzie dostępny pod adresem \url{localhost:8080/geoserver/web}. Standardowo nazwą użytkownika jest "admin" natomiast hasłem "geoserver".

Kolejnym krokiem jest podłączenie źródła danych z bazy PostgreSQL do serwera \textit{geoserver} zgodnie z poradnikiem ze strony \url{http://docs.geoserver.org/stable/en/user/gettingstarted/postgis-quickstart/index.html} 

Po tych krok możliwe jest pobranie całej mapy w formacie rastrowym do aplikacji przy użyciu usługi internetowej udostępnianej przez \textit{geoserver} - \textbf{WMS}.

Następnie zostaną stworzone trzy warstwy (ang. \textit{layers}) podstawie widoków wystawionych w bazie danych umożliwiające pobranie następujących danych w formacie GeoJSON:
\begin{itemize}
\item[--] dotacje od osób indywidualnych na kandydata w wyborach prezydenckich,
\item[--] dotacje od komitetów wyborczych na kandydata,
\item[--] połączenie powyższych dwóch wyników finansowych w celu uzyskania sumarycznej sumy pozyskanych środków przez konkretnego kandydata.
\end{itemize}

Odpowiednio widoki zostały stworzone na podstawie następujących wyrażeń:
\bigskip \noindent
Wpłaty indywidualne:
\begin{lstlisting}[style=BashInputStyle]
SELECT sum(cc.transaction_amt), cc.zip_code, gz.geom
FROM candidates ca
	inner join committees co on ca.cand_id=co.cand_id
	inner join ind_contribs cc on co.cmte_id=cc.cmte_id
	inner join geo_zip_codes gz on cc.zip_code=gz.zip_code 
WHERE ca.cand_id='%cand_id%' and ca.cand_office='P'
GROUP BY cc.zip_code, gz.geom
\end{lstlisting}
\bigskip \noindent
Wpłaty poprzez komitety wyborcze:
\begin{lstlisting}[style=BashInputStyle]
SELECT sum(cc.transaction_amt), cc.zip_code, gz.geom
FROM candidates ca
	inner join committees co on ca.cand_id=co.cand_id
	inner join ind_contribs cc on co.cmte_id=cc.cmte_id
	inner join geo_zip_codes gz on cc.zip_code=gz.zip_code 
WHERE ca.cand_id='%cand_id%' and ca.cand_office='P'
GROUP BY cc.zip_code, gz.geom
\end{lstlisting}
\bigskip \noindent
Łączne wpłaty:
\begin{lstlisting}[style=BashInputStyle]
SELECT coalesce(sum(ccc.ccsum + icc.icsum), 
	sum(ccc.ccsum), sum(icc.icsum)) as suma, 
	coalesce(ccc.zip_code, icc.zip_code) as zip_code, 
	coalesce(ccc.geom, icc.geom) as geom 
FROM
	(SELECT sum(cc.transaction_amt) as ccsum,
	 gz.zip_code as zip_code, gz.geom
	FROM candidates ca
		inner join committees co on ca.cand_id=co.cand_id
		inner join comm_contribs cc on co.cmte_id=cc.cmte_id
		inner join geo_zip_codes gz on cc.zip_code=gz.zip_code 
	WHERE ca.cand_office='P' and ca.cand_id='%cand_id%' 
	GROUP BY gz.zip_code, gz.geom) as ccc
FULL OUTER JOIN
	(SELECT sum(cc.transaction_amt) as icsum,
	 gz.zip_code as zip_code, gz.geom
	FROM candidates ca
		inner join committees co on ca.cand_id=co.cand_id
		inner join ind_contribs cc on co.cmte_id=cc.cmte_id
		inner join geo_zip_codes gz on cc.zip_code=gz.zip_code 
	WHERE ca.cand_office='P' and ca.cand_id='%cand_id%' 
	GROUP BY gz.zip_code, gz.geom) as icc
ON ccc.zip_code=icc.zip_code
GROUP BY coalesce(ccc.zip_code, icc.zip_code),
	 coalesce(ccc.geom, icc.geom)
\end{lstlisting}

Po operacji dodania warstw w panelu administracyjnym \textit{geoservera} możliwe jest wykonywanie zapytań do usługi WFS (Web Feature Service) umożliwiającej pobieranie danych geograficznych w postaci GeoJSON. Dane pobrane w takim formacie będzie można w odpowiedni sposób przeanalizować w aplikacji internetowej i stworzyć warstwy wektorowe umożliwiające prezentację danych na podkładzie mapy udostępnionej w ramach usług WMS.

\subsection{Przygotowanie aplikacji internetowej}
Ostatnim krokiem w celu prezentacji przechowywanych danych jest uruchomienie aplikacji internetowej. W przypadku naszej realizacji wykorzystaliśmy \textit{framwework} Express oparty o Node.JS. Komponenty HTML umożliwiające wybrania kandydata, czy źródła pobranych danych finansowych zostały zaimplementowane przy pomocy biblioteki AngularJS. 

Za prezentację danych geograficznych oraz wykonywanie zapytań do serwera \textit{geoserver} odpowiada biblioteka openlayers (zaimplementowana w języku JavaScript).

\bigskip \noindent
Aby uruchomić aplikację konieczne jest wykonanie następującej komendy:
\begin{lstlisting}[style=BashInputStyle]
  # npm start
\end{lstlisting}

Teraz domyślnie aplikacja będzie dostępna pod adresem \url{http://localhost:3000/}

\bigskip \noindent
\bigskip \noindent

\section{Załączniki}
Skrypt tworzący strukturę bazy danych:
\begin{lstlisting}[style=BashInputStyle]
DROP RULE IF EXISTS 
"CANDIDATES_on_duplicate_ignore" ON CANDIDATES;
DROP RULE IF EXISTS
 "COMMITTEES_on_duplicate_ignore" ON COMMITTEES;
DROP RULE IF EXISTS
 "IND_CONTRIBS_on_duplicate_ignore" ON IND_CONTRIBS;
DROP RULE IF EXISTS
 "COMM_CONTRIBS_on_duplicate_ignore" ON COMM_CONTRIBS;
DROP RULE IF EXISTS
 "COMM_CAND_LINKAGES_on_duplicate_ignore" ON COMM_CAND_LINKAGES;

DROP TABLE IF EXISTS COMM_CAND_LINKAGES;
DROP TABLE IF EXISTS COMM_CONTRIBS;
DROP TABLE IF EXISTS IND_CONTRIBS;
DROP TABLE IF EXISTS COMMITTEES;
DROP TABLE IF EXISTS CANDIDATES;

CREATE TABLE CANDIDATES(
    CAND_ID VARCHAR(9) PRIMARY KEY,
    CAND_NAME VARCHAR(200),
    CAND_ELECTION_YR NUMERIC(4),
    CAND_CITY VARCHAR(30),
    CAND_ST VARCHAR(2),
    CAND_ZIP VARCHAR(9),
    CAND_OFFICE_ST VARCHAR(2),
    CAND_OFFICE VARCHAR(1),
    CAND_OFFICE_DISTRICT VARCHAR(2),
    CAND_PTY_AFFILIATION VARCHAR(3)
);

CREATE TABLE COMMITTEES(
    CMTE_ID VARCHAR(9) PRIMARY KEY,
    CMTE_NM VARCHAR(200),
    CMTE_CITY VARCHAR(30),
    CMTE_ST VARCHAR(2),
    CMTE_ZIP VARCHAR(9),
    CAND_ID VARCHAR(9),
    CMTE_TP VARCHAR(1),
    CMTE_PTY_AFFILIATION VARCHAR(3),
    FOREIGN KEY (CAND_ID) REFERENCES CANDIDATES(CAND_ID)
);

CREATE TABLE COMM_CAND_LINKAGES(
    LINKAGE_ID VARCHAR(9) PRIMARY KEY,
    CAND_ID VARCHAR(9),
    CAND_ELECTION_YR NUMERIC(4),
    FEC_ELECTION_YR NUMERIC(4),
    CMTE_ID VARCHAR(9),
    CMTE_TP VARCHAR(1),
    CMTE_DSGN VARCHAR(1),
    FOREIGN KEY (CAND_ID) REFERENCES CANDIDATES (CAND_ID),
    FOREIGN KEY (CMTE_ID) REFERENCES COMMITTEES (CMTE_ID)
);

CREATE TABLE COMM_CONTRIBS(
    SUB_ID NUMERIC(19) PRIMARY KEY,
    CMTE_ID VARCHAR(9),
    ENTITY_TP VARCHAR(3),
    NAME VARCHAR(200),
    CITY VARCHAR(30),
    STATE VARCHAR(2),
    ZIP_CODE VARCHAR(9),
    CAND_ID VARCHAR(9),
    TRANSACTION_AMT NUMERIC(14,2),
    TRANSACTION_DT DATE,
    FOREIGN KEY (CAND_ID) REFERENCES CANDIDATES (CAND_ID),
    FOREIGN KEY (CMTE_ID) REFERENCES COMMITTEES (CMTE_ID)
);

CREATE TABLE IND_CONTRIBS(
    SUB_ID NUMBER(19) PRIMARY KEY,
    CMTE_ID VARCHAR(9),
    ENTITY_TP VARCHAR(3),
    NAME VARCHAR(200),
    CITY VARCHAR(30),
    STATE VARCHAR(2),
    ZIP_CODE VARCHAR(9),
    TRANSACTION_AMT NUMERIC(14,2),
    TRANSACTION_DT DATE,
    FOREIGN KEY (CMTE_ID) REFERENCES COMMITTEES (CMTE_ID)
);

CREATE INDEX ON COMMITTEES (CMTE_ZIP);
CREATE INDEX ON COMMITTEES (CAND_ID);

CREATE INDEX ON CANDIDATES (CAND_ELECTION_YR);
CREATE INDEX ON CANDIDATES (CAND_ZIP);

CREATE INDEX ON COMM_CAND_LINKAGES (CAND_ELECTION_YR);
CREATE INDEX ON COMM_CAND_LINKAGES (CMTE_ID);

CREATE INDEX ON COMM_CONTRIBS (ZIP_CODE);
CREATE INDEX ON COMM_CONTRIBS (CAND_ID);
CREATE INDEX ON COMM_CONTRIBS (TRANSACTION_DT);
CREATE INDEX ON COMM_CONTRIBS (TRANSACTION_AMT);

CREATE INDEX ON IND_CONTRIBS (ZIP_CODE);
CREATE INDEX ON IND_CONTRIBS (TRANSACTION_DT);
CREATE INDEX ON IND_CONTRIBS (TRANSACTION_AMT);

CREATE RULE "CANDIDATES_on_duplicate_ignore" 
AS ON INSERT TO CANDIDATES
  WHERE EXISTS(SELECT 1 FROM CANDIDATES
                WHERE (CAND_ID)=(NEW.CAND_ID))
  DO INSTEAD NOTHING;

CREATE RULE "COMMITTEES_on_duplicate_ignore"
 AS ON INSERT TO COMMITTEES
  WHERE EXISTS(SELECT 1 FROM COMMITTEES
              WHERE (CMTE_ID)=(NEW.CMTE_ID))
  DO INSTEAD NOTHING;

CREATE RULE "IND_CONTRIBS_on_duplicate_ignore"
 AS ON INSERT TO IND_CONTRIBS
  WHERE EXISTS(SELECT 1 FROM IND_CONTRIBS
              WHERE (SUB_ID)=(NEW.SUB_ID))
  DO INSTEAD NOTHING;

CREATE RULE "COMM_CONTRIBS_on_duplicate_ignore"
 AS ON INSERT TO COMM_CONTRIBS
  WHERE EXISTS(SELECT 1 FROM COMM_CONTRIBS
              WHERE (SUB_ID)=(NEW.SUB_ID))
  DO INSTEAD NOTHING;

CREATE RULE "COMM_CAND_LINKAGES_on_duplicate_ignore" 
AS ON INSERT TO COMM_CAND_LINKAGES
  WHERE EXISTS(SELECT 1 FROM COMM_CAND_LINKAGES
              WHERE (LINKAGE_ID)=(NEW.LINKAGE_ID))
  DO INSTEAD NOTHING;

\end{lstlisting}

\end{document}