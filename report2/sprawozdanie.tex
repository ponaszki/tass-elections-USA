\documentclass[10pt,a4paper]{article}
\usepackage[utf8]{inputenc}
\usepackage{polski}
\usepackage{amsmath}
\usepackage{amsfonts}
\usepackage{amssymb}
\usepackage{graphicx}
\usepackage{verbatim}
%\usepackage{minted}
\usepackage{amsmath}
\usepackage{algorithm}
\usepackage[noend]{algpseudocode}
\usepackage{csvsimple}
\usepackage{caption}
\usepackage{listings}     
\usepackage{subcaption}
\usepackage{hyperref}
\usepackage{url}
\usepackage[T1]{fontenc}
\usepackage{lmodern}
\usepackage[most]{tcolorbox}
\usepackage{xcolor}
\usepackage{listings}

\lstdefinestyle{BashInputStyle}{
  language=bash,
  basicstyle=\small\sffamily,
  numbers=left,
  numberstyle=\tiny,
  numbersep=3pt,
  frame=tb,
  columns=fullflexible,
  backgroundcolor=\color{yellow!20},
  linewidth=0.9\linewidth,
  xleftmargin=0.1\linewidth
}

\newcommand*{\Package}[1]{\texttt{#1}}%

\author{Onaszkiewicz Przemysław, Gadawski Łukasz}
\title{Temat 2\\ Dla wskazanego kandydata w wyborach w USA, przedstaw w sposób czytelny na mapie finansowanie jego kampanii wyborczej, z uwzględnieniem darowizn bezpośrednich i poprzez komitety wyborcze. 
}

\begin{document}
\maketitle

\section{Architektura rozwiązania}
Aplikacja składa się następujących elementów:
\begin{enumerate}
\item Baza danych PostgreSQL wraz z rozszerzeniem PostGIS.
\item Instacja serwera \textit{geoserver} umożliwiająca pobieranie danych geograficznych z bazy danych.
\item Zadania zaimplementowane jako tzw. \textit{task}i w systemie budowania wersji gradle:
\begin{itemize}
\item[--] \textit{getData} - umożliwiające pobranie plików CSV zawierających dane numeryczne odpowiednich danych finansowania. Poprzez zmianę skryptu możliwe jest pobranie danych z różnych przedziałów lat.
\item[--] \textit{cleanDb} - wykonuje połączenie z bazą danych oraz wykonanie skryptu tworzącego strukturę bazy danych.
\end{itemize} 
\item Skrypt w języku python przetwarzający pliki CSV z danymi dotyczącymi finansowania i ładującymi odpowiednie dane do bazy danych.
\item Aplikacja internetowa oparta o framework aplikacji internetowych \textit{Express} stworzony pod kątem aplikacji napisanych w Node.js.
\begin{itemize}
\item[--] biblioteka AngualarJS udostępniająca komponenty HTML,
\item[--] biblioteka openlayers umożliwiające prezentację danych pobranych z serwera \textit{geoserver},z
\end{itemize}
\end{enumerate}
\section{Opis instalacji}
Aplikacja była testowana na systemie \textit{Linux Mint 17.3 Cinnamon 64-bit} w wersji \textit{2.8.6}. 

\bigskip
Instalacja bazy danych wraz z rozszerzeniem PostGIS:
\begin{lstlisting}[style=BashInputStyle]
  # sudo apt-get update
  # sudo apt-get install postgresql postgresql-contrib 
  	postgis postgresql-9.4-postgis-2.1
  	postgresql-9.4-postgis-scripts 
  	postgresql-9.4-postgis-2.1-scripts
\end{lstlisting}

\bigskip
Przygotowanie bazy danych pod kątek wykorzystania przez serwer \textit{geoserver} oraz aplikację internetową. Stworzenie użytkownika bazy danych:
\begin{lstlisting}[style=BashInputStyle]
  # sudo -i -u postgres
  # createuser --interactive // create "tass-user"
\end{lstlisting}

\bigskip
Stworzenie bazy danych:
\begin{lstlisting}[style=BashInputStyle]
  # createdb tass
\end{lstlisting}

\bigskip
Instalacja rozszerzenie PostGIS umożliwiającego wykonywanie operacji na danych geograficznych:
\begin{lstlisting}[style=BashInputStyle]
  # sudo -i -u postgres // login as superuser
  # psql -d tass
  # CREATE EXTENSION postgis;
  # CREATE EXTENSION postgis_topology;
\end{lstlisting}


\end{document}